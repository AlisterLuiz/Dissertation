% Chapter Template

\chapter{Introduction} % Main chapter title

\label{ChapterX} % Change X to a consecutive number; for referencing this chapter elsewhere, use \ref{ChapterX}

%----------------------------------------------------------------------------------------
%	SECTION 1
%----------------------------------------------------------------------------------------

COVID-19, which was declared a global pandemic by the World Health Organization on March 11, 2020, has affected
millions of lives worldwide in terms of both health and finances and have also had 
a severe global economic impact.

With over a million tests carried out on average daily across the world for diagnosing patients with COVID-19 \cite{OWD2020}, 
it is therefore the need of the hour to alleviate the burden on healthcare professionals who conducts these diagnoses on a day-to-day basis, 
and more importantly, minimize the exposure rate between patients and healthcare professionals.
\section{Aim}

This project aims to \textbf{automate the diagnoses of COVID-19 with medical imagery using Deep Learning}. The main focus being
to achieve the highest diagnostic accuracy possible, minimizing the false-negative rate, which, if not accounted for may have adverse real-world implications. 
We also intend to compare the usage of X-rays vs CT scans in this project considering 
the limitations in terms of the equipment available on medical facilities, radiation exposure, and the results obtained in both cases.

The overall goal of this project is to develop a framework that enables highly accurate rapid 
diagnoses of COVID-19. This would lead to a safer environment 
for healthcare professionals by minimizing the rate of exposure and following quarantine protocols 
therefore curbing the spread of COVID-19.
% %-----------------------------------
% %	SUBSECTION 1
% %-----------------------------------
% \subsection{Subsection 1}

% Nunc posuere quam at lectus tristique eu ultrices augue venenatis. Vestibulum ante ipsum primis in faucibus orci luctus et ultrices posuere cubilia Curae; Aliquam erat volutpat. Vivamus sodales tortor eget quam adipiscing in vulputate ante ullamcorper. Sed eros ante, lacinia et sollicitudin et, aliquam sit amet augue. In hac habitasse platea dictumst.

% %-----------------------------------
% %	SUBSECTION 2
% %-----------------------------------

% \subsection{Subsection 2}
% Morbi rutrum odio eget arcu adipiscing sodales. Aenean et purus a est pulvinar pellentesque. Cras in elit neque, quis varius elit. Phasellus fringilla, nibh eu tempus venenatis, dolor elit posuere quam, quis adipiscing urna leo nec orci. Sed nec nulla auctor odio aliquet consequat. Ut nec nulla in ante ullamcorper aliquam at sed dolor. Phasellus fermentum magna in augue gravida cursus. Cras sed pretium lorem. Pellentesque eget ornare odio. Proin accumsan, massa viverra cursus pharetra, ipsum nisi lobortis velit, a malesuada dolor lorem eu neque.

%----------------------------------------------------------------------------------------
%	SECTION 2
%----------------------------------------------------------------------------------------

\section{Objectives}

The objective of this project is to develop a framework 
that rapidly diagnoses patients with COVID-19 with medical imagery and thereby provide assistance to healthcare professionals.

These are the primary objectives for this thesis:
\begin{enumerate}
    \item Analyse X-ray and CT imaging features of COVID-19. 
    \item Build a deep learning model that diagnoses patients with COVID-19 using chest X-rays. 
    \item Test the accuracy of the proposed COVID-19 diagnoses model by comparing with other similar models implemented previously. 
    \item Develop a deep learning API that can be used by healthcare professionals and medical facilities to diagnose COVID-19.
    \item Optionally, build a deep learning model that diagnoses patients with COVID-19 using chest CT scans and compare the accuracy and results obtained from both models respectively.
  \end{enumerate}

\section{Manuscript Organization}

This manuscript contains 4 chapters, starting with a comprehensive \textbf{Introduction} 
of the main objectives of this project. This is followed by the 
\textbf{Literature Review} chapter which aims to synthesize and sum up relevant research and implementations previously conducted in this same field. 
The next chapter \textbf{Requirements Analysis} conducts a detailed study on the use cases of this project and identifies user requirements and labels their priority. An additional section \textbf{Design} 
has also been included, which demonstrates the pipeline and the workflow of the proposed model.
Following this, is a section for \textbf{Evaluation Strategy} which specifies the analysis and assessment that needs to be administered for this project. The last chapter is dedicated to \textbf{Project Management} which provides a detailed schedule that must be strictly adhered to, in order to ensure the success of this project, as well as examining the risks involved and the ethical, legal and, social issues pertaining to this project.