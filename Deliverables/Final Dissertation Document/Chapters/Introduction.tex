% Chapter Template

\chapter{Introduction} % Main chapter title

\label{ChapterX} % Change X to a consecutive number; for referencing this chapter elsewhere, use \ref{ChapterX}

%----------------------------------------------------------------------------------------
%	SECTION 1
%----------------------------------------------------------------------------------------

COVID-19, which was declared a global pandemic by the World Health Organization on March 11, 2020, has affected
millions of lives worldwide in terms of both health and finances and have also had 
a severe global economic impact.

Over a billion tests have been carried across the world for diagnosing patients with COVID-19 \cite{STA21}, 
it is therefore the need of the hour to alleviate the burden on healthcare professionals who conducts these diagnoses on a day-to-day basis, 
and more importantly, minimize the exposure rate between patients and healthcare professionals.
\section{Aim}

This project aims to \textbf{automate the diagnoses of COVID-19 with medical imagery using Deep Learning}. The main focus being
to achieve the highest diagnostic accuracy possible, minimizing the false-negative rate, which, if not accounted for may have adverse real-world implications. 

The overall goal of this project is to develop an automated workflow that enables highly accurate rapid 
diagnoses of COVID-19 for both symptomatic and asymptomatic patients. This would lead to a safer environment 
for healthcare professionals by minimizing the rate of exposure 
therefore minimizing the spread of COVID-19.

\section{Objectives}

The objective of this project is to develop a deep learning classification model 
that rapidly diagnoses patients with COVID-19 using medical imagery and thereby provide assistance to healthcare professionals.

These are the primary objectives for this thesis:
\begin{enumerate}
    \item Analyse X-ray imaging features of COVID-19. 
    % and CT 
    \item Build a deep learning model that diagnoses patients with COVID-19 using chest X-rays. 
    \item Test the accuracy of the proposed COVID-19 diagnoses model by comparing with other similar models implemented previously. 
    \item  Develop a deep learning API that can be used by healthcare professionals and medical facilities to diagnose COVID-19.
    \item Optionally, build a deep learning model that diagnoses patients with COVID-19 using chest CT scans and compare the accuracy and results obtained from both models respectively.
  \end{enumerate}
  
\section{Extra Achievements}

In addition to the above objectives, we proposed an ensemble learning approach for COVID-19 diagnosis that achieved a higher accuracy than all the papers implemented for both X-ray and CT scans reported in this manuscript. We have also drafted a research paper on the experiments conducted for X-ray classification. We intend to publish it to the \textit{Computers in Biology and Medicine} journal and is attached in Appendix \ref{appendix:paper}. Furthermore, we have also validated our results with the help of a senior Radiologist who has provided us his findings on a sample of X-ray and CT scans.

\section{Manuscript Organization}

This manuscript contains 5 chapters, starting with a comprehensive \textbf{Introduction} 
of the main objectives of this project. This is followed by the 
\textbf{Literature Review} chapter which aims to synthesize and sum up relevant research and implementations previously conducted in this same field. The \textbf{Project Implementation} chapter covers the dataset collection, pre-processing, and methodology used to implement this project. Following this, \textbf{Results and Evaluation} is where we showcase the results obtained and conduct a critical evaluation of our findings. The last chapter \textbf{Conclusion and Future Works} summarizes our experiments, provides an insight into the major challenges faced, and future works that we plan to carry out.