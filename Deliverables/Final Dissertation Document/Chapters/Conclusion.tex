\chapter{Conclusion and Future Work} \label{Conclusion and Future Works}

This chapter concludes this manuscript and comments on the limitations of this project, future works we wish to undertake and finally my thoughts and reflections on the experiences throughout this year-long project.

\section{Limitations}

Despite our best efforts there are indeed some limitations to this project implementation which could be overcome in future iterations. The first being the availability of a reasonably sized dataset. We have noticed that there have only been a handful of researchers and institutions actively collecting chest X-rays and CT scans with most of the data not being open-source. Therefore, most of the datasets on Kaggle and other forums usually cite the same root source. Although we have performed Data Augmentation to generate more samples, we believe having more COVID-19 samples would indeed increase the reliability of our model.

The free-to-use tier of Google Colab was another major limitation in training our models. Due to the limited RAM (12GB) and runtime duration (12 hours per day), we had to incorporate memory optimization strategies such as reducing the batch size and image dimensions in order to be able to fit the data into memory and conduct training. This was especially for the case of CT scans. Furthermore, we also had a ceiling for the number of augmented images generated due to the same memory limitation.

As we have utilized the free credit offered by Microsoft Azure for students, the Ubuntu Virtual Machine used to host our Python Scripts and Flutter Web Application on their platform would only run for around 2.5 months, after which a monthly fee of \$30 apply for continued use. 

\section{Future Work}

We have a lot of future plans for this project. Provided we gain access to a powerful machine well beyond the capabilities of a standard Google Colab environment, we believe we can further increase the reliability and performance of our models. Here are a list of proposed ideas:
\begin{enumerate}
    \item \textbf{Data Sources} - Identify more data sources to increase the size of our existing dataset.
    \item \textbf{Image Dimensions} - Given a machine with higher RAM, perform model training with larger image dimensions.
    \item \textbf{Data Augmentation} - Utilize Data Augmentation to generate more samples per class.
    \item \textbf{Multi-class CT Classification} - Obtain CT scans from Pneumonia patients and build multi-class classification model.
    \item \textbf{Research Papers} - Publish research papers on our X-ray and CT scan implementation, along with a survey paper.
    \item \textbf{Radiologist Validation} - Validate more model results and heatmaps with a Radiologist and receive their feedback on the web application and thereby make suitable changes.
\end{enumerate}
\section{Reflections}

Through this year-long project, I was able to gain valuable experience and improve both technical and soft skills through various phases of this project. More importantly, being my first ever major Data Science project, I was able to learn from my mistakes and overcome all the roadblocks faced with the guidance and support of my project supervisor, Dr. Hani Ragab, who was always available to clear my doubts, provide fruitful insights and resources. 

I would also like to thank Dr. Hadj Batatia and Professor Kayvan Karim for providing their thoughts and assistance towards this project. Lastly, I would also like to take a moment to thank my parents for their constant prayers and motivation throughout the project, who pushed me to achieve my goals and become the person I am today. 

I truly believe that my commitment and hard work coupled with the guidance and encouragement provided by my project supervisor, professors, and parents have led to this project achieve its potential and successful completion.