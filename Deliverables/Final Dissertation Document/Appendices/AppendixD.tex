% Appendix Template

\chapter{COVID-19 Lung Characteristics} \label{COVID-19 Lung Characteristics} % Change X to a consecutive letter; for referencing this appendix elsewhere, use \ref{AppendixX}


\section*{Additional Study on COVID-19 Patients}

Morales et al. conducted a systematic literature review with meta-analysis 
of the imaging features of COVID-19 also observed similar lung characteristics 
such as GGO's from X-ray scans from patients diagnosed with COVID-19 from 
multiple studies \cite{RAJ+2020}, the results are tabulated in Table \ref{tab:Chest X-ray Review Results}.
\vspace{1em}

\begin{longtable}{| p{.27\textwidth} | p{.20\textwidth} | p{.20\textwidth} | p{.20\textwidth} |} 
    
    \hline
\textbf{Study} & \textbf{Unilateral Pneumonia} & \textbf{Bilateral Pneumonia} & \textbf{Ground-glass Opacity} \\
\hline

Huang et al. \cite{CYX+2020} & - & 40 (97.6\%) & 12 (29.3\%)\\ \hline
Chen et al. \cite{CNM+2020} & 25 (25.3\%) & 74 (74.7\%) & 14 (14.1\%)\\ \hline
Wang et al. \cite{WDC+2020} & 0 (0\%)& 138 (100.0\%) & 138 (100.0\%)\\ \hline
Liu et al. \cite{LKY+2020} & - & 36 (26.3\%) & 55 (40.1\%)\\ \hline
Chang et al. \cite{CML+2020} & 1 (7.7\%)& - & 6 (46.2\%)\\ \hline
Pan et al. \cite{PYH+2020} & - & 38 (60.3\%) & 14 (22.2\%)\\ \hline
Zhang et al. \cite{ZMX+2020} & 2 (22.2\%)& 5 (55.6\%)& 7 (77.8\%)\\ \hline

 \caption{Chest X-ray imaging characteristics from multiple studies}

    \label{tab:Chest X-ray Review Results}
    \end{longtable}
\vspace{-1em}